\documentclass{article}
\usepackage[margin=1in]{geometry}
\usepackage[utf8]{inputenc}
\usepackage{hyperref}
\usepackage{courier}
\usepackage{textcomp}
\usepackage{indentfirst}

\title{SMDO group environment setup guideline}
\author{Yicong Fu}
\date{August 2021}

\begin{document}

\maketitle

This document covers steps to install the software developed by the group on
a linux machine (Ubuntu 20.04 LTS).

\section{Prerequisite software and libraries}

The following list contains the basic software and libraries you'll need
to have in your machine. Note that for lab's desktops you will not have the
sudo (superuser) privilege, so you'll need to contact IT support
(support@ae.gatech.edu) to install them for you.

\begin{itemize}
    \item GNU Make (\texttt{sudo apt install make})
    \item CMake (\texttt{sudo apt install cmake})
    \item GNU C++ compiler (\texttt{sudo apt install g++})
    \item MPI C++ compiler (\texttt{sudo apt install openmpi-bin})
    \item Linear algebra libraries: BLAS/LAPACK

    (\texttt{sudo apt install libblas-dev liblapack-dev})

    \item Python package installer: (\texttt{sudo apt install python3-pip})
    \item numpy, mpi4py, Cython: (\texttt{pip3 install numpy mpi4py Cython})
    \item Paraview (\texttt{sudo apt install paraview})

\end{itemize}
\section{TACS}

This sections shows how to install the finite element code TACS.

\begin{itemize}
    \item Repository: \href{https://github.com/smdogroup/tacs}{https://github.com/smdogroup/tacs}
    \item Documentation: \href{https://smdogroup.github.io/tacs/}{https://smdogroup.github.io/tacs/}
\end{itemize}

\subsection{Installation}

\begin{enumerate}
    \item Clone the repository. The group's convention is to download it to
    \texttt{\$HOME/git/}:

    \texttt{git clone git@github.com:smdogroup/tacs.git}

    \item Download and install METIS:

    \texttt{cd \$HOME/git/tacs/extern}

    \texttt{wget http://glaros.dtc.umn.edu/gkhome/fetch/sw/metis/metis-5.1.0.tar.gz}

    \texttt{tar -xzf metis-5.1.0.tar.gz}

    \texttt{cd metis-5.1.0}

    \texttt{make config prefix=\$HOME/git/tacs/extern/metis/ CFLAGS="-fPIC"}

    \texttt{make install}

    \item Create \texttt{Makefile.in} file (and customize it, if needed):

    \texttt{cd \$HOME/git/tacs \&\& cp Makefile.in.info Makefile.in}

    \item Compile TACS and python wrapper:

    \texttt{make \&\& make interface}

    \item Set up \texttt{\$PYTHONPATH} for the python wrapper, add the following line to \texttt{\texttildelow/.bashrc}:

    \texttt{export PYTHONPATH=\$\string{PYTHONPATH\string}:\texttildelow/git/tacs}

    \item Then, run \texttt{source \texttildelow/.bashrc}

\end{enumerate}

\subsection{Test}

Examples in \texttt{tacs/examples} can be used to test if TACS (and python wrapper) was built
successfully. For example, you can run

\texttt{cd \texttildelow/git/tacs/examples/triangle \&\& python3 triangle.py}

if it runs, then TACS has been installed properly.

\section{ParOpt}

This section shows how to intall ParOpt, the parallel gradient-based optimizer.

\begin{itemize}
    \item Repository: \href{https://github.com/smdogroup/paropt}{https://github.com/smdogroup/paropt}
    \item Documentation: \href{https://smdogroup.github.io/paropt/}{https://smdogroup.github.io/paropt/}
\end{itemize}

\subsection{Dependencies}

ParOpt would need MPI, numpy, mpi4py, cython and BLAS/LAPACK. But if you already
installed TACS then no extra packages are needed.

\subsection{Installation}

\begin{enumerate}

    \item Clone the repository. The group's convention is to download it to
    \texttt{\$HOME/git/}:

    \texttt{git clone git@github.com:smdogroup/paropt.git}

    \item Create \texttt{Makefile.in} file (and customize it, if needed):

    \texttt{cp Makefile.in.info Makefile.in}

    \item Compile ParOpt and python wrapper:

    \texttt{cd \$HOME/git/paropt \&\& make \&\& make interface}

    \item Set up \texttt{\$PYTHONPATH} for the python wrapper, add the following line to \texttt{\texttildelow/.bashrc}:

    \texttt{export PYTHONPATH=\$\string{PYTHONPATH\string}:\texttildelow/git/paropt}

    \item Then, run \texttt{source \texttildelow/.bashrc}

\end{enumerate}

\subsection{Test}

You could run \texttt{\texttildelow/git/paropt/examples/rosenbrock/rosenbrock.py} to test if the installation succeeded.

\section{TMR}

This section covers steps to install TMR, the mesh refinement tool.

\begin{itemize}
    \item Repository: \href{https://github.com/smdogroup/tmr}{https://github.com/smdogroup/tmr}
    \item Documentation: \href{https://smdogroup.github.io/tmr/}{https://smdogroup.github.io/tmr/}
\end{itemize}

\subsection{Download the code}

\begin{enumerate}
    \item
    Clone the repository. The group's convention is to download it to \texttt{\$HOME/git/}:

    \texttt{git clone git@github.com:smdogroup/tmr.git}

    \item
    Create and modify \texttt{Makefile.in} file:

    \texttt{cp Makefile.in.info Makefile.in}

    Then comment out the last three lines regarding the netgen libraries.

\end{enumerate}

\subsection{Dependencies}

You need to install the following dependencies before installing TMR.

\begin{itemize}
    \item TACS, ParOpt

    \item Blossom V

    \begin{enumerate}
        \item Download the package to \texttt{tmr/extern}:

        \texttt{cd \$HOME/git/tmr/extern}

        \texttt{wget https://pub.ist.ac.at/\texttildelow vnk/software/blossom5-v2.05.src.tar.gz}

        \texttt{tar -zxf blossom5-v2.05.src.tar.gz}

        \item Compile:

        \texttt{cd blossom5-v2.05.src}

        \texttt{cp ../Makefile-blossom5 Makefile \&\& make \&\& make lib}

    \end{enumerate}

    \item OpenCASCADE

    \begin{enumerate}
        \item Create the directory for packages:

        \texttt{mkdir \$HOME/packages \&\& cd \$HOME/packages}

        \item Download OpenCASCADE:

        \texttt{wget https://acdl.mit.edu/ESP/OCC741lin64.tgz}

        \texttt{tar -zxf OCC741lin64.tgz}

        \texttt{mv OpenCASCADE-7.4.1/ OpenCASCADE}

        \item Add the following lines to \texttt{\texttildelow/.bashrc}

        \texttt{export CASROOT=\$HOME/packages/OpenCASCADE}

        \texttt{export CASARCH=}

        \texttt{export PATH=\$CASROOT/bin:\$PATH}

        \texttt{export LD\string_LIBRARY\string_PATH=\$LD\string_LIBRARY\string_PATH:\$CASROOT/lib}

        \item Then, run \texttt{source \texttildelow/.bashrc}

        \item Add the following values to \texttt{TMR\string_DEBUG\string_FLAGS} and
        \texttt{TMR\string_FLAGS} in \texttt{\$HOME/git/tmr/Makefile.in}:

        \texttt{-DTMR\string_HAS\string_OPENCASCADE -DTMR\string_HAS\string_EGADS
        -DTMR\string_HAS\string_PAROPT}


    \end{enumerate}

    \item egads4py

    \begin{enumerate}
        \item Clone the repository to \texttt{\$HOME/git}:

        \texttt{cd \$HOME/git \&\& git clone git@github.gatech.edu:gkennedy9/egads4py.git}

        \item Copy over \texttt{Makefile.in.info} to \texttt{Makefile.in}

        \item Compile:

        \texttt{make}
        \texttt{make interface}

        \item Set up \texttt{\$PYTHONPATH} for the python wrapper,
        add the following line to \texttt{\texttildelow/.bashrc}:

        \texttt{export PYTHONPATH=\$\string{PYTHONPATH\string}:\texttildelow/git/egads4py}

        \item Then, run \texttt{source \texttildelow/.bashrc}

    \end{enumerate}

\end{itemize}


\subsection{Installation}

\begin{enumerate}

    \item Add the following lines to \texttt{\$\texttildelow/.bashrc}:

    \texttt{export TACS\string_DIR=\$HOME/git/tacs}

    \texttt{export PAROPT\string_DIR=\$HOME/git/paropt}

    \texttt{export EGADS\string_DIR=\$HOME/git/egads4py}

    \item Then, run \texttt{source \texttildelow/.bashrc}

    \item Compile the code and python wrapper:

    \texttt{cd \$HOME/git/tmr \&\& make \&\& make interface}

    \item Set up \texttt{\$PYTHONPATH} for the python wrapper,
    add the following line to \texttt{\texttildelow/.bashrc}:

    \texttt{export PYTHONPATH=\$\string{PYTHONPATH\string}:\texttildelow/git/tmr}

    \item Then, run \texttt{source \texttildelow/.bashrc}

\end{enumerate}

\subsection{Test}

You could run \texttt{\texttildelow/git/tmr/examples/egads/cylinder/cylinder.py} to test if the installation succeeded.

\section{Miscellaneous}

Some commands and resources that might be useful.

\begin{itemize}
    \item Remote access to lab's desktop

    You can use the following command to ssh into your PC in the office:

    \texttt{ssh -J your-GTID@ssh.ae.gatech.edu your-GTID@AE-SMDO-XXXXXX.ae.gatech.edu}

    where \texttt{XXXXXX} is the machine number that can be found on your PC case.

    \item PACE training series

    PACE has a series of training sessions on git, linux, and PACE cluster (that has thousands of nodes
    and is we run very large jobs).
    If you want to get yourself familiar with those topics, you could
    check the following website for more details:

    \href{https://pace.gatech.edu/training}{https://pace.gatech.edu/training}

    \item The-Missing-Semester-Of-Your-CS-Education

    A short course by MIT that covers many useful topics about the tools for developers,
    including shell, command line environment, git, and so on:

    \href{https://missing.csail.mit.edu/}{https://missing.csail.mit.edu/}

\end{itemize}




\end{document}
